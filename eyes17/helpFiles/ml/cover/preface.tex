\documentclass[12pt,a4paper]{report}
\usepackage{graphicx}
\usepackage[utf8]{inputenc}
%%%%%%%%%%%%%%%%%%%%%%%%%%%%%%%%%%%%%%%%%%%%%%%%%%%%%%%%%%%%%%%%
% LaTeX code  to enable                                        %
% the build by XeLaTeX, specific for Malayalam language        %
%%%%%%%%%%%%%%%%%%%%%%%%%%%%%%%%%%%%%%%%%%%%%%%%%%%%%%%%%%%%%%%%
\usepackage{pdfpages}
\usepackage{fontspec}
\usepackage{polyglossia,xltxtra}
\setdefaultlanguage{malayalam}
\setmainfont[Script=Malayalam, HyphenChar="00AD]{Rachana}
\newfontfamily\engfont[Mapping=tex-text]{FreeSerif}
\newfontfamily\malayalamfontsf{Rachana}[Script=Malayalam]
\newfontfamily\malayalamfonttt{Rachana}[Script=Malayalam]
\def\xromn#1{{\fontencoding{T1}\fontfamily{cmr}\selectfont#1}}
\def\squote{\xromn{`}}
\def\esquote{\xromn{'}}
\def\dquote{\xromn{``}}
\def\edquote{\xromn{''}}
\DeclareTextFontCommand{\myfont}{\engfont}
\def\mdash{\myfont{\textemdash}}
\def\ndash{\myfont{\textendash}}
\lefthyphenmin=3
\righthyphenmin=4
\linespread{1.2}
\widowpenalty=10000
\clubpenalty=10000
\raggedbottom
\sloppy
%%%%%%%%%%%%%%%%%%%%%%%%%%%%%%%%%%%%%%%%%%%%%%%%%%%%%%%%%%%%%%%%
%                end of inserted code                          %
%%%%%%%%%%%%%%%%%%%%%%%%%%%%%%%%%%%%%%%%%%%%%%%%%%%%%%%%%%%%%%%%

\begin{document}
\thispagestyle{empty}


അവതാരിക 

കമ്പ്യൂട്ടറുമായി ഘടിപ്പിക്കാവുന്ന ഉപകരണങ്ങൾ ഉപയോഗിച്ച് സയൻസ് പരീക്ഷണങ്ങൾ നടത്തുന്ന രീതി ഇന്ത്യൻ സർവകലാശാലകളിലെ വിദ്യാർത്ഥികൾക്കു പരിചയപ്പെടുത്തുക എന്ന ഉദ്ദേശത്തോടെ 2004ൽ ദൽഹി ആസ്ഥാനമായി പ്രവർത്തിക്കുന്ന ഇന്റർ യൂണിവേഴ്സിറ്റി ആക്സിലറേറ്റർ സെന്റർ എന്ന സ്ഥാപനം PHOENIX എന്ന പേരിൽ ഒരു പദ്ധതി ആരംഭിച്ചു.  ലളിതവും നിർമാണച്ചെലവ് കുറഞ്ഞതുമായ ഉപകരണങ്ങൾ വികസിപ്പിക്കുക, അധ്യാപകർക്ക് അതിൽ പരിശീലനം നൽകുക എന്ന രണ്ടു ലക്ഷ്യങ്ങൾ വെച്ചാണ് ഇതാരംഭിച്ചത്. ഉപകരണത്തിന്റെ വില ഒരു വിദ്യാർത്ഥിക്ക് പോലും താങ്ങാനാവുന്നതായിരിക്കണം എന്നതിനാൽ ഉപകരണങ്ങൾ താരതമ്യേന ലളിതമാക്കാൻ ശ്രമിച്ചട്ടുണ്ട്. കോളേജുകളിലെ പരീക്ഷണശാലകളുടെ സമയപരിധികളിൽ നിന്നും താല്പര്യമുള്ള വിദ്യാർഥികളെയെങ്കിലും മോചിപ്പിക്കുക എന്നൊരുദ്ദേശ്ശവും ഉണ്ടായിരുന്നു. ഇതിന്റെ രൂപകല്പനകൾ സ്വതന്ത്രമായി ആർക്കും ലഭ്യമാണ്.

സോഫ്റ്റ്‌വെയർ GNU ജനറൽ പബ്ലിക് ലൈസൻസിലും ഹാർഡ്‌വെയർ  CERN ഓപ്പൺ ഹാർഡ്‌വെയർ ലൈസൻസിലുമാണ് ലഭ്യമാക്കുന്നത്. ഈ പ്രോജെക്ടിൽ നിന്നുള്ള ഏറ്റവും പുതിയ ഉത്പന്നമായ ExpEYES-17 ലഭ്യമാക്കുന്നതിൽ പലർക്കും പങ്കുണ്ട്. ഈ പ്രോജെക്ടിനെ മുൻപോട്ടു കൊണ്ടുപോകുന്നതിൽ പ്രധാന പങ്കുവഹിച്ച അധ്യാപക,വിദ്യാർത്ഥി സമൂഹത്തോടൊപ്പം  ജിതിൻ ബി പി രൂപപ്പെടുത്തിയ ഈ ഉപകരണത്തെ PHOENIXനു വേണ്ടി ലഭ്യമാക്കിയതിൽ  IUAC ഡയറക്ടർ Dr. D. Kanjilal വഹിച്ച പങ്കിനും  ഞങ്ങൾ നന്ദി രേഖപ്പെടുത്തുന്നു.

ഈ ഗ്രന്ഥത്തിന്റെ പതിപ്പുകൾ GNU ജനറൽ ഡോക്യൂമെന്റഷൻ ലൈസൻസിൽ വിതരണം ചെയ്യാവുന്നതാണ്.

അജിത്കുമാർ ബി പി 
വി വി വി സത്യനാരായണ 
http://expeyes.in

\end{document}
