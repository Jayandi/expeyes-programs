\documentclass[12pt,a4paper]{report}
\usepackage{graphicx}
\usepackage[utf8]{inputenc}
\DeclareUnicodeCharacter{00A0}{\nobreakspace}
\usepackage[T1]{fontenc}
\usepackage[english]{babel}
\usepackage{times}
\begin{document}
\thispagestyle{empty}

Preface

The PHOENIX (\textsc{Physics with Home-made Equipment \& Innovative
Experiments}) project was started in 2004 by \textsc{Inter- University
Accelerator Centre} with the objective of improving the science
education at Indian Universities. Development of low cost laboratory
equipment and training teachers are the two major activities under
this project.

\textsc{expEYES-17} is an advanced version of \textsc{expEYES}
released earlier. It is meant to be a tool for learning by
exploration, suitable for high school classes and above. We have tried
optimizing the design to be simple, flexible, rugged and low cost.
The low price makes it affordable to individuals and we hope to see
students performing experiments outside the four walls of the
laboratory, that closes when the bell rings.

The software is released under \textsc{GNU General Public License} and
the hardware under \textsc{CERN Open Hardware Licence.} The project
has progressed due to the active participation and contributions from
the user community and many other persons outside \textsc{IUAC}. We
are thankful to Dr D Kanjilal for taking necessary steps to obtain
this new design from its developer Jithin B P, CSpark Research.

\textsc{expEYES-17} user's manual is distributed under \textsc{GNU
Free Documentation License}.



Ajith Kumar B.P. ~~~~~~~~~(ajith@iuac.res.in) ~~http://expeyes.in

V V V Satyanarayana
\end{document}
